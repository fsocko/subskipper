%%--SubSkipper Documentation
%%--Modified 25/07/15

\documentclass{article}

%%--packages
\usepackage{hyperref}
\usepackage{color}
\usepackage{graphicx}
\graphicspath{{../figures/}}
\usepackage{geometry}
\usepackage[table]{xcolor}
\usepackage{tabularx}
\usepackage{amssymb}
\usepackage{graphicx}

%%--Define Margins
 \geometry{
 a4paper,
 total={170mm,257mm},
 left=20mm,
 right=20mm,
 top=20mm,
 bottom=20mm,
 }


%%--commands
\newcommand{\degree}{$^{\circ}$}
%%tabularX centered and expanding column
\newcolumntype{Y}{>{\centering\arraybackslash}X}

%%--Define some new arrows - description from base to arrowhead
\newcommand{\aUR}{\Rsh} %Arrow Up Right
\newcommand{\aLU}{\rotatebox[origin=c]{90}{$\Rsh$}} %Arrow left Up
\newcommand{\aDL}{\rotatebox[origin=c]{180}{$\Rsh$}} %arrown down left
\newcommand{\aRD}{\rotatebox[origin=c]{270}{$\Rsh$}} %%a right down
\newcommand{\aUL}{\Lsh} %%a up left
\newcommand{\aLD}{\rotatebox[origin=c]{90}{$\Lsh$}} %%a left down
\newcommand{\aDR}{\rotatebox[origin=c]{180}{$\Lsh$}} %a Down Right
\newcommand{\aRU}{\rotatebox[origin=c]{270}{$\Lsh$}} %a right up

%%Array format for showing conversion factors.
%%\conversion{unit left}{top factor}{bottom factor}{unit right}
%\conversion{m}{12}{15}{NM}
\newcommand{\conversion}[4]{
$\begin{array}{rcl}
\aUR & \times {#2} & \aRD \\
\textrm{{#1}} &  & \textrm{{#4}} \\
\aLU & \times {#3} &  \aDL \\
\end{array}$
\vspace{12pt}
}


\author{fSocko}
\title{Pacific Campaign and Torpedo Solution Reference (v1.1)}

\begin{document}

\maketitle
\pagebreak
\tableofcontents
\pagebreak

\section{Introduction}

The main purpose of this documentation is to record techniques and methods of early submarine attack techniques in a way which are simple to employ in computer programs (i.e. showing mathematical equations where possible), as well as acting as a reference for Submarine Simulators.
\\ \\
This document is a reference for Torpedo types and common conversion factors, as well as types of US Fleet Boats and Types of US Torpedoes. A brief recognition manual will be provided in an appendix. This document also aims to provide a brief overview of common torpedo solution methods.

\section{Unit Conversion Factors}
\vspace{12pt}
\begin{center}
\setlength{\extrarowheight}{-2pt}
\begin{large}
\begin{tabularx}{0.95\textwidth}{Y  Y}
\textbf{Length} & \textbf{Speed} \\

\\
\conversion{NM}{1852.00}{0.00054}{m} 
& \conversion{NM/h}{1}{1}{kt}   \\

\conversion{NM}{2025.37}{0.00049}{yd}
& \conversion{m/s}{1.94384}{0.514444}{kt} \\
 
\conversion{NM}{6076.12}{0.00016}{ft}
& \conversion{yd/s}{1.77745}{0.562603}{kt} \\ 

\conversion{Mi}{1609.34}{0.00062}{m}
& \conversion{ft/s}{0.592484}{1.68781}{kt} \\

\conversion{Mi}{1760}{0.00057}{yd} & \\
\conversion{ m}{1.09361}{0.914}{yd} & \\
\conversion{ m}{3.28084}{0.3048}{ft} & \\
%%Speed

\end{tabularx}
\end{large}
\end{center}

\subsection{Speed}
\includegraphics[width=\textwidth]{fixedWire}

\subsection{Nomogram (Metric)}
\vspace{24pt}
\includegraphics[height=\textwidth, angle=90]{nomo-metric}
\label{nomo-metric}
\pagebreak

\section{Torpedo Data}
%%TODO: German Torpedoes
The following is torpedo data for German torpedo types as used in the European Theatre of Operations.
Slow and Fast speeds and ranges removed as appropriate.
%Adapted from:
%https://en.wikipedia.org/wiki/G7e_torpedo
%LSH3 - Torpedo pullout
%http://silenthunterubf.forumfree.it/?t=51767849
%http://uboat.net/technical/torpedoes.htm

\subsubsection{TI (G7a)}
\begin{tabular}{l|l}
Available from:& Always\\
Range(Slow):& 12.5km\\
Range(Medium):& 7.5km\\
Range(Fast):& 6km\\
Speed(Slow):&30kt\\
Speed(Medium):&40kt\\
Speed(Fast):&44kt\\
Powerplant:&Steam\\
\end{tabular} \\
Notes:\\
The T1 was very reliable and possessed good range. These torpedoes left a trail of bubbles behind them as they ran. Thus should be used only for night or long range attacks.

\subsubsection{TII (G7e)}
\begin{tabular}{l|l}
Available from:& Always\\
Range(Slow):& 5km\\
Speed(Slow):& 30kt\\
Powerplant:&Electric\\
\end{tabular} \\
Notes:\\
Both of its detonators were flawed. The magnetic influence mechanism, often would detonate prematurely, or not at all. Estimates of the failure rate of T2 torpedoes for one reason or another range between 20\% and 40\%. Problems fixed with the TIII after Norwegian campaign (June 1940).

\subsubsection{TIIIa (G7e)}
\begin{tabular}{l|l}
Available from:& June 1942\\
Range(Slow):& 7.5km\\ %%Check if it's not 5000m
Speed(Slow):& 30kt\\
Powerplant:&Electric\\
\end{tabular} \\
Notes:\\
The T3 represented a vast improvement over the early T2. The faulty exploders from the T2 were scrapped in favor of a new design.

\subsubsection{TIV (G7es) Falke}
\begin{tabular}{l|l}
Available from:& July 1943\\
Range(Slow):& 7.5km\\ %%Check if it's not 5.7km
Speed(Slow):& 20kt\\
Powerplant:&Electric\\
\end{tabular} \\
Notes:\\
Acoustic homing torpedo. 
Slow speed and no magnetic pistol. The sensor was sensitive to targets moving at between 12 and 19 knots. The arming range was shortened to 250 meters. Occasionally the torpedo would turn around and try to sink the U-boat thus, when launched the U-boat should dive to 60m.


\subsubsection{TV (G7es) Zaunk\"onig I}
\begin{tabular}{l|l}
Available from:& October 1943\\
Powerplant:&Electric\\
Range(Slow):& 5.7km\\
Speed(Slow):& 24kt\\
\end{tabular} \\
Notes:\\
Acoustic homing torpedo. Active after a straight run of 400m. Warhead of 274kg.


\subsubsection{TXI (G7es) Zaunk\"onig II}
\begin{tabular}{l|l}
Available from:& July 1944\\
Powerplant:&Electric\\
Range(Slow):& 5.7km\\
Speed(Slow):& 24kt\\
\end{tabular} \\
Notes:\\
Acoustic homing torpedo. Active after a straight run of 400 metres. A modified TV, less affected by Foxer.

\subsection{Warheads}
All the German U-boat torpedoes were 53.3cm (21 inch) in diameter and had a warhead of 280kg (The T5 had 274kg).

\subsection{FAT}
Available March 1943. The FAT (Federapparat Torpedo) ran a wandering course with regular 180-degree turns. Useful against convoys, and was fitted to both G7a and G7e T3s. It runs up and down on parallel legs of 800 or 1600 m length.

\subsection{LUT}
Available March 1944. LUT Lageunabh\"angiger Torpedo (bearing independent torpedo), changes the torpedo's course to a preset heading directly after launch, so the u-boot can fire such torpedoes at targets without changing its own course.


%%TODO:\section{ETO - UBoot Types}

%%Maths and shit - Make this an input, it stays the same.

%% Contents of original AngriffScheibe Handbuch which are relevant w/out the disk:
%Example 4: Determine the speed of a target by plotting position over time
%Example 5: Determine the speed of a target knowing the target length
%Example 6: Find the speed of a target with constant relative bearing
%EG. 7: Find the speed of a target with a changing relative bearing
%Example 8: Find the speed of a target with two bearing and range observations
%Example 9: Find the AOB of a target with the Aspect Ratio method
%Example 10: Determine the distance to the track of a target
%Example 11: Determine an optimum speed for attack position
%12: Determine the lead angle for a perpendicular attack with 0° gyro angle
%Example 13: Determine the torpedo gyro angle

%TODO

%Example 6: Find the speed of a target with constant relative bearing
%EG. 7: Find the speed of a target with a changing relative bearing
%Example 8: Find the speed of a target with two bearing and range observations
%Example 9: Find the AOB of a target with the Aspect Ratio method


\section{Calculating AOB Based on Aspect Ratio}

%Methods adapted from Angriffscheibe Handbuch by Karl Hahn, 2008.

AOB can be determined given the following data:
\begin{itemize}
\item{Observed Mast Height}
\item{Observed Ship Length}
\item{Reference Aspect Ratio (i.e. $\frac{Reference Length}{Reference Mast Height}$)}
\end{itemize}

\subsection{Determine an observed \emph{aspect ratio}.}
$$AR\_{observed} = \frac{Observed Length}{Observed Mast Height}$$
As the required figure is a ratio, it does not matter in what units the figures are given. For example, this could be the number of degrees Length and Mast Heigh subtend, the number of periscope graduations subtended or angular length in metres. It only matters that units for Observed Mast Height and Observed Length are the same.

\subsubsection{Determine the Reference Aspect Ratio}
Identify the target and find the Length and Mast Height as given in the recognition manual (if possible) or calculate these figures using the SubSkipper ship parser. Proceed as for the observed aspect ratio to get the Reference Aspect Ratio (\emph{ARreference}).

\subsection{AOB calculation}
$$AOB = arcSin \frac{AR\_{observed}}{AR\_{reference}}$$

\begin{itemize}
\item{Note: This method is less accurate as AOB approaches 0. Using this method with a sample size of 16 at various AOBs, the average error was 9.1\degree and the median error 6.88\degree. The optimum range for collecting data seems to be around 2000m.}
\item{Note: This method does not compute whether the AOB is on the port or starboard side.}
\item{Note: "The AOB can only go up to 90, and gives no indication of starboard or port side showing. You have to determine that visually.}

\item{
AOB (Relative to Target) for each Quadrant:
	\begin{itemize}
	\item{0\degree - 90\degree : Use AOB result as is.}
	\item{90\degree - 180\degree : Subtract AOB result from 180.}
	\item{180\degree - 270\degree : Subtract AOB result from 180.}
	\item{270\degree - 360\degree : Subtract AOB result from 360.}
	\end{itemize}

}
\end{itemize}

\section{Determine the distance to the track of a target}

As you manoeuvre into a firing position, it is useful to know your distance to the track of your target. Knowing the range and AOB of the target:

Track: intersection point between Submarine course and the current target course.

$$Distance to Track = Range \left( sin(AOB) \right)$$
%
%\section{Determine an optimum speed for attack position}
%
%To determine own speed needed to reach the target track in time to set up a torpedo solution. (Requires: Target Speed, Target AOB, Bearing to Target, Target Range, Distance to target track)
%
%\begin{enumerate}
%\item{Find distance T must travel to reach a bearing of 000\degree. If the track is known (see previous step) and bearing and range to target is known, time required is:
%$$ s = d/t  $$
%}


%Example 11: Determine an optimum speed for attack position


\section{Determine the lead angle for a perpendicular attack with 0\degree gyro angle (AKA O\'Kane Solution)}

Once you are turned onto a perpendicular intercept course, this method calculates when to fire. The lead angle is simply the target bearing at the moment you fire. The range is irrelevant to the calculation although it is best to plan the attack within 500-1000m.

$$ lead angle = tan^{-1} ( [target speed / torpedo speed] ) $$

\section{Determine the torpedo gyro angle}
This example will show how to calculate the required gyro angle for a firing solution. This method is somewhat simplified and is best for smaller gyro angles (\textless 30\degree) and shorter ranges (\textless 1000 metres).

$$sin (lead angle) / sin (AOB) = (target speed) / (torpedo speed)$$
$$ gyroAngle = targetBearing -– leadAngle $$

\section{O'Kane Table of Values}
\centering{\includegraphics[height=10 cm]{OKaneValues}}

\pagebreak

\section{Torpedo spread at range (m) per angle of spread (\degree)}
\centering{\includegraphics[width=\textwidth]{SpreadAtRangeBW}}



\input{knotToKM}

\includegraphics[angle=90, width=\textwidth]{cmapETO}
\label{GWX Campaign Map}
\includegraphics[angle=90, width=\textwidth]{ETOFlags}

%\begin{figure}
%\includegraphics[angle=90, width=\textwidth]{cmapPAC}
%\label{Pacific Campaign Map}
%\end{figure}

%\section{Recognition Manual (Short, Metric)}
%%TODO: Include section heading in the rec manual and make sure that the textHeight fits a page.

%TODO: ETO recog man
%%\centering
\begin{tabularx}{\textwidth}{|r|l|l|l|l|X|}
\hline
\textbf{Name} & \textbf{Speed (kn)} & \textbf{Length (m)} & \textbf{Mast (m)} & \textbf{Draft (m)} & \textbf{Aspect}\\
\hline
 Auxilary Subchaser & $17.00$ & $63.00$ & $11.40$ & $4.20$ & $5.5263$ \\
\hline
Agano Light Cruiser & $35.00$ & $175.00$ & $23.60$ & $5.70$ & $7.4153$ \\
\hline
Akikaze Destroyer & $34.00$ & $102.50$ & $21.80$ & $2.90$ & $4.7018$ \\
\hline
Akita Maru & $13.50$ & $105.00$ & $16.00$ & $7.30$ & $6.5625$ \\
\hline
Akitsu Escort Carrier & $20.00$ & $143.75$ & $35.70$ & $7.84$ & $4.0266$ \\
\hline
Akizuki Destroyer & $33.00$ & $135.00$ & $22.80$ & $3.80$ & $5.9211$ \\
\hline
Armed Merchant Cruiser & $17.50$ & $155.00$ & $26.80$ & $8.70$ & $5.7836$ \\
\hline
Armed Trawler & $12.00$ & $62.00$ & $13.50$ & $4.20$ & $4.5926$ \\
\hline
Asashio Destroyer & $35.00$ & $118.00$ & $27.00$ & $3.40$ & $4.3704$ \\
\hline
Atami Gunboat & $20.00$ & $45.00$ & $7.80$ & $1.10$ & $5.7692$ \\
\hline
Attacker Escort Carrier & $18.00$ & $152.00$ & $33.80$ & $8.20$ & $4.4970$ \\
\hline
Baltimore Heavy Cruiser & $33.00$ & $205.00$ & $30.60$ & $6.50$ & $6.6993$ \\
\hline
Biyo Maru & $12.00$ & $120.00$ & $20.10$ & $7.50$ & $5.9701$ \\
\hline
Black Swan Sloop & $19.50$ & $91.30$ & $27.20$ & $3.50$ & $3.3566$ \\
\hline
Bogue Escort Carrier & $18.00$ & $152.00$ & $11.30$ & $8.20$ & $13.4513$ \\
\hline
British Medium Old Tanker & $10.00$ & $91.50$ & $11.00$ & $4.50$ & $8.3182$ \\
\hline
Brooklyn Light Cruiser & $32.50$ & $185.00$ & $32.30$ & $7.00$ & $5.7276$ \\
\hline
Buckley Destroyer Escort & $24.00$ & $93.20$ & $21.10$ & $3.40$ & $4.4171$ \\
\hline
Buzyun Maru & $11.00$ & $103.00$ & $9.80$ & $6.40$ & $10.5102$ \\
\hline
Casablanca Escort Carrier & $19.00$ & $156.00$ & $13.90$ & $6.30$ & $11.2230$ \\
\hline
Chitose Seaplane Tender & $28.90$ & $192.50$ & $32.60$ & $7.50$ & $5.9049$ \\
\hline
Clemson Destroyer & $35.00$ & $95.00$ & $28.80$ & $3.10$ & $3.2986$ \\
\hline
Cleveland Light Cruiser & $32.50$ & $185.00$ & $36.20$ & $6.60$ & $5.1105$ \\
\hline
Coastal Composite Freighter & $13.00$ & $80.77$ & $17.90$ & $5.48$ & $4.5123$ \\
\hline
Colorado Battleship & $21.00$ & $190.20$ & $43.90$ & $9.10$ & $4.3326$ \\
\hline
Conte Verde Liner & $21.00$ & $173.74$ & $28.60$ & $7.50$ & $6.0748$ \\
\hline
Dido Light Cruiser & $32.00$ & $156.00$ & $36.50$ & $6.30$ & $4.2740$ \\
\hline
Elco Torpedo Boat & $44.00$ & $24.70$ & $5.30$ & $1.20$ & $4.6604$ \\
\hline
Elite Black Swan Sloop & $19.50$ & $91.30$ & $27.20$ & $3.50$ & $3.3566$ \\
\hline
Etorofu Escort & $19.70$ & $66.00$ & $17.90$ & $2.93$ & $3.6872$ \\
\hline
Evarts Destroyer Escort & $19.00$ & $89.50$ & $27.70$ & $3.40$ & $3.2310$ \\
\hline
Fiji Light Cruiser & $32.00$ & $169.00$ & $43.90$ & $5.00$ & $3.8497$ \\
\hline
Fishing boat & $10.00$ & $38.00$ & $18.70$ & $1.80$ & $2.0321$ \\
\hline
Fishing boat & $10.00$ & $38.00$ & $18.70$ & $1.80$ & $2.0321$ \\
\hline
Fishing Boat & $8.00$ & $34.00$ & $7.90$ & $2.40$ & $4.3038$ \\
\hline
Fishing Trawler & $12.00$ & $62.00$ & $13.50$ & $4.20$ & $4.5926$ \\
\hline
Fleet Carrier & $32.50$ & $250.00$ & $54.40$ & $6.80$ & $4.5956$ \\
\hline
Fletcher Destroyer & $35.00$ & $115.00$ & $28.00$ & $4.90$ & $4.1071$ \\
\hline
Flower Corvette & $16.00$ & $60.60$ & $19.70$ & $3.50$ & $3.0761$ \\
\hline
Fubuki Destroyer & $38.00$ & $118.50$ & $22.60$ & $3.50$ & $5.2434$ \\
\hline
Furutaka Heavy Cruiser & $33.00$ & $188.00$ & $25.90$ & $5.20$ & $7.2587$ \\
\hline
Fuso Battleship & $24.70$ & $212.00$ & $48.30$ & $9.50$ & $4.3892$ \\
\hline
Hakusika Maru & $12.00$ & $135.00$ & $20.30$ & $8.70$ & $6.6502$ \\
\hline
Haruna Maru & $11.00$ & $76.00$ & $19.70$ & $4.60$ & $3.8579$ \\
\hline
Heito Maru & $17.00$ & $103.60$ & $16.60$ & $6.50$ & $6.2410$ \\
\hline
Hira Gunboat & $20.00$ & $55.00$ & $10.10$ & $1.10$ & $5.4455$ \\
\hline
Hiryu Fleet Carrier & $34.30$ & $223.30$ & $37.20$ & $7.50$ & $6.0027$ \\
\hline
Horai Maru & $17.00$ & $137.46$ & $25.60$ & $8.53$ & $5.3695$ \\
\hline
Iowa Battleship & $33.00$ & $270.40$ & $44.30$ & $11.00$ & $6.1038$ \\
\hline
Ise Battleship & $25.60$ & $220.00$ & $44.50$ & $9.50$ & $4.9438$ \\
\hline
Ise Battleship (Late War) & $25.60$ & $220.00$ & $49.20$ & $9.50$ & $4.4715$ \\
\hline
J Class Destroyer & $36.00$ & $108.50$ & $28.00$ & $4.60$ & $3.8750$ \\
\hline
JC Butler Destroyer Escort & $24.00$ & $93.30$ & $28.10$ & $2.80$ & $3.3203$ \\
\hline
Junk & $10.00$ & $19.60$ & $17.70$ & $1.20$ & $1.1073$ \\
\hline
Kasagisan Maru & $12.50$ & $86.80$ & $18.10$ & $6.10$ & $4.7956$ \\
\hline
\end{tabularx}
\pagebreak

\begin{tabularx}{\textwidth}{|r|l|l|l|l|X|}
\hline
\textbf{Name} & \textbf{Speed (kn)} & \textbf{Length (m)} & \textbf{Mast (m)} & \textbf{Draft (m)} & \textbf{Aspect}\\
\hline
Kent Heavy Cruiser & $31.50$ & $183.00$ & $21.80$ & $5.00$ & $8.3945$ \\
\hline
King George V Battleship & $27.00$ & $244.00$ & $53.90$ & $9.10$ & $4.5269$ \\
\hline
Kinposan Maru & $14.50$ & $108.00$ & $18.50$ & $6.60$ & $5.8378$ \\
\hline
Kisaragi Destroyer & $37.00$ & $102.00$ & $11.20$ & $3.00$ & $9.1071$ \\
\hline
Kiturin Maru & $18.50$ & $130.00$ & $25.70$ & $8.60$ & $5.0584$ \\
\hline
Kongo Battleship & $30.50$ & $220.00$ & $42.50$ & $8.50$ & $5.1765$ \\
\hline
Kuma Light Cruiser & $36.00$ & $163.00$ & $35.60$ & $6.10$ & $4.5787$ \\
\hline
Landing Ship Tank & $10.80$ & $98.70$ & $25.20$ & $3.30$ & $3.9167$ \\
\hline
Large German Tanker & $18.00$ & $190.00$ & $24.20$ & $10.70$ & $7.8512$ \\
\hline
Large Sampan & $10.00$ & $35.00$ & $26.90$ & $2.20$ & $1.3011$ \\
\hline
Liberty Cargo & $13.00$ & $147.00$ & $21.00$ & $5.80$ & $7.0000$ \\
\hline
Maya Heavy Cruiser & $34.00$ & $204.00$ & $33.50$ & $5.90$ & $6.0896$ \\
\hline
Medium Modern Passenger/Freighter & $16.00$ & $105.00$ & $27.40$ & $7.30$ & $3.8321$ \\
\hline
Minekaze Destroyer & $39.00$ & $102.60$ & $21.80$ & $2.90$ & $4.7064$ \\
\hline
Mogami Heavy Cruiser & $35.00$ & $200.00$ & $35.00$ & $6.70$ & $5.7143$ \\
\hline
Momi Destroyer & $18.00$ & $102.60$ & $11.10$ & $2.93$ & $9.2432$ \\
\hline
Momoyama Maru & $11.00$ & $120.00$ & $16.00$ & $4.70$ & $7.5000$ \\
\hline
Momoyama Maru & $11.00$ & $120.00$ & $16.00$ & $4.70$ & $7.5000$ \\
\hline
Mutsuki Destroyer & $37.00$ & $102.00$ & $11.20$ & $3.00$ & $9.1071$ \\
\hline
Nagara Maru & $19.00$ & $137.00$ & $20.50$ & $7.50$ & $6.6829$ \\
\hline
Naka Light Cruiser & $35.25$ & $162.00$ & $31.30$ & $5.10$ & $5.1757$ \\
\hline
Nevada Battleship & $20.50$ & $190.20$ & $43.90$ & $9.10$ & $4.3326$ \\
\hline
New Mexico Battleship & $22.00$ & $190.00$ & $42.00$ & $10.00$ & $4.5238$ \\
\hline
Nippon Maru & $20.00$ & $150.00$ & $16.70$ & $8.60$ & $8.9820$ \\
\hline
North Carolina Battleship & $27.00$ & $220.00$ & $33.40$ & $9.00$ & $6.5868$ \\
\hline
Northampton Heavy Cruiser & $32.70$ & $183.00$ & $49.00$ & $7.00$ & $3.7347$ \\
\hline
Okinoshima Minelayer & $20.00$ & $123.40$ & $28.50$ & $4.80$ & $4.3298$ \\
\hline
Old Raked Bow Split Merchant & $11.00$ & $120.00$ & $17.30$ & $4.70$ & $6.9364$ \\
\hline
Omaha Light Cruiser & $35.00$ & $168.00$ & $19.50$ & $4.50$ & $8.6154$ \\
\hline
Patrol Boat 102 & $35.00$ & $95.00$ & $28.70$ & $3.10$ & $3.3101$ \\
\hline
Pennsylvania Battleship & $21.00$ & $190.20$ & $43.90$ & $9.10$ & $4.3326$ \\
\hline
Pocket Battleship & $28.50$ & $183.00$ & $37.90$ & $7.00$ & $4.8285$ \\
\hline
River class Frigate & $20.00$ & $91.50$ & $25.90$ & $4.00$ & $3.5328$ \\
\hline
Sampan & $10.00$ & $19.50$ & $16.50$ & $1.20$ & $1.1818$ \\
\hline
Schnellboat & $44.00$ & $24.70$ & $5.30$ & $1.20$ & $4.6604$ \\
\hline
Shimushu Escort & $20.00$ & $70.00$ & $24.50$ & $2.93$ & $2.8571$ \\
\hline
Shiratsuyu Destroyer & $34.00$ & $108.00$ & $22.00$ & $3.00$ & $4.9091$ \\
\hline
Shokaku Fleet Carrier & $34.50$ & $250.00$ & $32.40$ & $8.70$ & $7.7160$ \\
\hline
Small Old Split Freighter & $17.00$ & $80.50$ & $17.90$ & $3.70$ & $4.4972$ \\
\hline
Small Split Freighter & $12.50$ & $86.87$ & $18.10$ & $6.10$ & $4.7994$ \\
\hline
Somers Destroyer & $37.00$ & $116.00$ & $31.50$ & $4.20$ & $3.6825$ \\
\hline
Submarine Tender & $18.00$ & $201.00$ & $22.40$ & $9.40$ & $8.9732$ \\
\hline
T3 Tanker & $18.00$ & $190.00$ & $25.50$ & $10.70$ & $7.4510$ \\
\hline
Taiho Fleet Carrier & $33.30$ & $253.70$ & $46.20$ & $9.50$ & $5.4913$ \\
\hline
Taihosan Maru & $13.00$ & $80.50$ & $17.90$ & $3.70$ & $4.4972$ \\
\hline
Taiyo Escort Carrier & $21.00$ & $180.10$ & $15.10$ & $7.50$ & $11.9272$ \\
\hline
Takao Heavy Cruiser & $36.50$ & $204.00$ & $35.30$ & $6.00$ & $5.7790$ \\
\hline
Tennessee Battleship & $20.50$ & $190.00$ & $42.00$ & $10.00$ & $4.5238$ \\
\hline
Tennessee Battleship & $20.50$ & $190.20$ & $43.90$ & $9.10$ & $4.3326$ \\
\hline
Tribal Destroyer & $36.50$ & $115.00$ & $15.40$ & $4.00$ & $7.4675$ \\
\hline
Troop Transport & $16.00$ & $152.00$ & $23.20$ & $8.50$ & $6.5517$ \\
\hline
Tug Boat & $15.00$ & $63.00$ & $11.50$ & $4.20$ & $5.4783$ \\
\hline
Tyohei Maru & $13.00$ & $79.00$ & $14.30$ & $4.60$ & $5.5245$ \\
\hline
Type C Escort & $19.00$ & $89.50$ & $27.70$ & $3.40$ & $3.2310$ \\
\hline
Type D Escort & $19.00$ & $89.50$ & $27.70$ & $3.40$ & $3.2310$ \\
\hline
\end{tabularx}
\pagebreak

\begin{tabularx}{\textwidth}{|r|l|l|l|l|X|}
\hline
\textbf{Name} & \textbf{Speed (kn)} & \textbf{Length (m)} & \textbf{Mast (m)} & \textbf{Draft (m)} & \textbf{Aspect}\\
\hline
V.W Destroyer & $34.00$ & $95.00$ & $21.00$ & $3.20$ & $4.5238$ \\
\hline
Victory Cargo & $17.00$ & $139.60$ & $21.80$ & $7.30$ & $6.4037$ \\
\hline
Wakatake Patrol Boat & $18.00$ & $102.60$ & $21.70$ & $2.93$ & $4.7281$ \\
\hline
West Virginia Battleship & $21.00$ & $190.00$ & $42.00$ & $10.00$ & $4.5238$ \\
\hline
Yamato Battleship & $27.40$ & $263.00$ & $39.20$ & $10.80$ & $6.7092$ \\
\hline
Yugumo Destroyer & $35.00$ & $118.00$ & $11.50$ & $3.40$ & $10.2609$ \\
\hline
Z Class Destroyer & $38.00$ & $118.50$ & $22.60$ & $3.50$ & $5.2434$ \\
\hline
Zinbu Maru & $10.00$ & $120.00$ & $18.60$ & $4.70$ & $6.4516$ \\
\hline
\end{tabularx}
\pagebreak



%%%END END END
\pagebreak
\begin{thebibliography}{9}

\bibitem{angrHandB}
  Karl Hahl,
  Kriegsmarine Angriffscheibe Handbuch,
  p. 15,
  2008.
  
\bibitem{tvreAcqData}
http://www.tvre.org/en/acquiring-torpedo-firing-data,
Acquiring torpedo firing data,
2015.

\bibitem{okane90Angles}
Corey "Gutted" Hardwell,
Silent Hunter IV 90\degree - AOB Firing Angles,
circ. 2008(?).

\bibitem{subsim}
http://www.subsim.com

\bibitem{arcSin}
MathOnWeb,
%http://mathonweb.com/help_ebook/html/functions_2.htm,
The arcsin function,
Accessed 17.08.2015,


\end{thebibliography}




\end{document}

