%%--SubSkipper Documentation
%%--Modified 25/07/15

\documentclass{article}

%%--packages
\usepackage{hyperref}
\usepackage{listings}
\usepackage{color}
\usepackage{graphicx}
\graphicspath{{figures/}}
\usepackage{geometry}

%%--Define Margins
 \geometry{
 a4paper,
 total={170mm,257mm},
 left=20mm,
 right=20mm,
 top=20mm,
 bottom=20mm,
 }


%%--Define style for Java code
\definecolor{dkgreen}{rgb}{0,0.6,0}
\definecolor{gray}{rgb}{0.5,0.5,0.5}
\definecolor{mauve}{rgb}{0.58,0,0.82}

\lstset{frame=tb,
  language=Java,
  aboveskip=3mm,
  belowskip=3mm,
  showstringspaces=false,
  columns=flexible,
  basicstyle={\small\ttfamily},
  numbers=none,
  numberstyle=\tiny\color{gray},
  keywordstyle=\color{blue},
  commentstyle=\color{dkgreen},
  stringstyle=\color{mauve},
  breaklines=true,
  breakatwhitespace=true,
  tabsize=3
}

%%--commands
\newcommand{\degree}{$^{\circ}$}

\author{Filip Socko}
\title{SubSkipper - Program Documentation}

\begin{document}

\maketitle
%%\tableofcontents


\section{The Program Documentation for SubSkipper}

\section{CoreLogic}

\subsection{aspectAOB(int estAOB, Ship target, double mastObs, double lenObs))}
Method for determining the AOB based on an observed and reference aspect ratio.
Note, this method requires an approximate visual AOB or it will return wildly
inaccurate figures, because that's how trigonometry works.
public class AspectAOB {
	
//Wrapper for calculateAOB, calculates AOB bearing with AOB estimate checking against
//a visual estimate of AOB due to the limits of this method.
	
//Errors: -1: NaN found in observed AR
// 		  -2: Observed aspect > Ref Aspect

public double aspectAOB(int estAOB, Ship target, double mastObs, double lenObs)


\begin{itemize}
\item{}
\end{itemize}









\end{document}

